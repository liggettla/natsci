\documentclass[]{book}
\usepackage{lmodern}
\usepackage{amssymb,amsmath}
\usepackage{ifxetex,ifluatex}
\usepackage{fixltx2e} % provides \textsubscript
\ifnum 0\ifxetex 1\fi\ifluatex 1\fi=0 % if pdftex
  \usepackage[T1]{fontenc}
  \usepackage[utf8]{inputenc}
\else % if luatex or xelatex
  \ifxetex
    \usepackage{mathspec}
  \else
    \usepackage{fontspec}
  \fi
  \defaultfontfeatures{Ligatures=TeX,Scale=MatchLowercase}
\fi
% use upquote if available, for straight quotes in verbatim environments
\IfFileExists{upquote.sty}{\usepackage{upquote}}{}
% use microtype if available
\IfFileExists{microtype.sty}{%
\usepackage{microtype}
\UseMicrotypeSet[protrusion]{basicmath} % disable protrusion for tt fonts
}{}
\usepackage[margin=1in]{geometry}
\usepackage{hyperref}
\hypersetup{unicode=true,
            pdftitle={On Matters Of Nature and Science},
            pdfauthor={L A Liggett},
            pdfborder={0 0 0},
            breaklinks=true}
\urlstyle{same}  % don't use monospace font for urls
\usepackage{natbib}
\bibliographystyle{apalike}
\usepackage{color}
\usepackage{fancyvrb}
\newcommand{\VerbBar}{|}
\newcommand{\VERB}{\Verb[commandchars=\\\{\}]}
\DefineVerbatimEnvironment{Highlighting}{Verbatim}{commandchars=\\\{\}}
% Add ',fontsize=\small' for more characters per line
\usepackage{framed}
\definecolor{shadecolor}{RGB}{248,248,248}
\newenvironment{Shaded}{\begin{snugshade}}{\end{snugshade}}
\newcommand{\KeywordTok}[1]{\textcolor[rgb]{0.13,0.29,0.53}{\textbf{#1}}}
\newcommand{\DataTypeTok}[1]{\textcolor[rgb]{0.13,0.29,0.53}{#1}}
\newcommand{\DecValTok}[1]{\textcolor[rgb]{0.00,0.00,0.81}{#1}}
\newcommand{\BaseNTok}[1]{\textcolor[rgb]{0.00,0.00,0.81}{#1}}
\newcommand{\FloatTok}[1]{\textcolor[rgb]{0.00,0.00,0.81}{#1}}
\newcommand{\ConstantTok}[1]{\textcolor[rgb]{0.00,0.00,0.00}{#1}}
\newcommand{\CharTok}[1]{\textcolor[rgb]{0.31,0.60,0.02}{#1}}
\newcommand{\SpecialCharTok}[1]{\textcolor[rgb]{0.00,0.00,0.00}{#1}}
\newcommand{\StringTok}[1]{\textcolor[rgb]{0.31,0.60,0.02}{#1}}
\newcommand{\VerbatimStringTok}[1]{\textcolor[rgb]{0.31,0.60,0.02}{#1}}
\newcommand{\SpecialStringTok}[1]{\textcolor[rgb]{0.31,0.60,0.02}{#1}}
\newcommand{\ImportTok}[1]{#1}
\newcommand{\CommentTok}[1]{\textcolor[rgb]{0.56,0.35,0.01}{\textit{#1}}}
\newcommand{\DocumentationTok}[1]{\textcolor[rgb]{0.56,0.35,0.01}{\textbf{\textit{#1}}}}
\newcommand{\AnnotationTok}[1]{\textcolor[rgb]{0.56,0.35,0.01}{\textbf{\textit{#1}}}}
\newcommand{\CommentVarTok}[1]{\textcolor[rgb]{0.56,0.35,0.01}{\textbf{\textit{#1}}}}
\newcommand{\OtherTok}[1]{\textcolor[rgb]{0.56,0.35,0.01}{#1}}
\newcommand{\FunctionTok}[1]{\textcolor[rgb]{0.00,0.00,0.00}{#1}}
\newcommand{\VariableTok}[1]{\textcolor[rgb]{0.00,0.00,0.00}{#1}}
\newcommand{\ControlFlowTok}[1]{\textcolor[rgb]{0.13,0.29,0.53}{\textbf{#1}}}
\newcommand{\OperatorTok}[1]{\textcolor[rgb]{0.81,0.36,0.00}{\textbf{#1}}}
\newcommand{\BuiltInTok}[1]{#1}
\newcommand{\ExtensionTok}[1]{#1}
\newcommand{\PreprocessorTok}[1]{\textcolor[rgb]{0.56,0.35,0.01}{\textit{#1}}}
\newcommand{\AttributeTok}[1]{\textcolor[rgb]{0.77,0.63,0.00}{#1}}
\newcommand{\RegionMarkerTok}[1]{#1}
\newcommand{\InformationTok}[1]{\textcolor[rgb]{0.56,0.35,0.01}{\textbf{\textit{#1}}}}
\newcommand{\WarningTok}[1]{\textcolor[rgb]{0.56,0.35,0.01}{\textbf{\textit{#1}}}}
\newcommand{\AlertTok}[1]{\textcolor[rgb]{0.94,0.16,0.16}{#1}}
\newcommand{\ErrorTok}[1]{\textcolor[rgb]{0.64,0.00,0.00}{\textbf{#1}}}
\newcommand{\NormalTok}[1]{#1}
\usepackage{longtable,booktabs}
\usepackage{graphicx,grffile}
\makeatletter
\def\maxwidth{\ifdim\Gin@nat@width>\linewidth\linewidth\else\Gin@nat@width\fi}
\def\maxheight{\ifdim\Gin@nat@height>\textheight\textheight\else\Gin@nat@height\fi}
\makeatother
% Scale images if necessary, so that they will not overflow the page
% margins by default, and it is still possible to overwrite the defaults
% using explicit options in \includegraphics[width, height, ...]{}
\setkeys{Gin}{width=\maxwidth,height=\maxheight,keepaspectratio}
\IfFileExists{parskip.sty}{%
\usepackage{parskip}
}{% else
\setlength{\parindent}{0pt}
\setlength{\parskip}{6pt plus 2pt minus 1pt}
}
\setlength{\emergencystretch}{3em}  % prevent overfull lines
\providecommand{\tightlist}{%
  \setlength{\itemsep}{0pt}\setlength{\parskip}{0pt}}
\setcounter{secnumdepth}{5}
% Redefines (sub)paragraphs to behave more like sections
\ifx\paragraph\undefined\else
\let\oldparagraph\paragraph
\renewcommand{\paragraph}[1]{\oldparagraph{#1}\mbox{}}
\fi
\ifx\subparagraph\undefined\else
\let\oldsubparagraph\subparagraph
\renewcommand{\subparagraph}[1]{\oldsubparagraph{#1}\mbox{}}
\fi

%%% Use protect on footnotes to avoid problems with footnotes in titles
\let\rmarkdownfootnote\footnote%
\def\footnote{\protect\rmarkdownfootnote}

%%% Change title format to be more compact
\usepackage{titling}

% Create subtitle command for use in maketitle
\newcommand{\subtitle}[1]{
  \posttitle{
    \begin{center}\large#1\end{center}
    }
}

\setlength{\droptitle}{-2em}

  \title{On Matters Of Nature and Science}
    \pretitle{\vspace{\droptitle}\centering\huge}
  \posttitle{\par}
    \author{L A Liggett}
    \preauthor{\centering\large\emph}
  \postauthor{\par}
      \predate{\centering\large\emph}
  \postdate{\par}
    \date{2019-04-08}

\usepackage{booktabs}
\usepackage{amsthm}
\makeatletter
\def\thm@space@setup{%
  \thm@preskip=8pt plus 2pt minus 4pt
  \thm@postskip=\thm@preskip
}
\makeatother

\begin{document}
\maketitle

{
\setcounter{tocdepth}{1}
\tableofcontents
}
\chapter{Prerequisites}\label{prerequisites}

This is a \emph{sample} book written in \textbf{Markdown}. You can use
anything that Pandoc's Markdown supports, e.g., a math equation
\(a^2 + b^2 = c^2\).

The \textbf{bookdown} package can be installed from CRAN or Github:

\begin{Shaded}
\begin{Highlighting}[]
\KeywordTok{install.packages}\NormalTok{(}\StringTok{"bookdown"}\NormalTok{)}
\CommentTok{# or the development version}
\CommentTok{# devtools::install_github("rstudio/bookdown")}
\end{Highlighting}
\end{Shaded}

Remember each Rmd file contains one and only one chapter, and a chapter
is defined by the first-level heading \texttt{\#}.

To compile this example to PDF, you need XeLaTeX. You are recommended to
install TinyTeX (which includes XeLaTeX):
\url{https://yihui.name/tinytex/}.

\chapter{Genetics and Genomics}\label{g2}

floobity

\section{Introduction}\label{introduction}

A haplotype block is a set of closely linked alleles or markers on a
chromosome that tend to be inherited together over evolutionary time.

Across Eukaryotes, the frequency of recombination is inversely
proportional to overall genome size. The result is that yeast have a
recombination rate a few orders of magnitude higher than that of humans
(She and Jarosz, 2018).

\subsection{Subpoint}\label{subpoint}

This is some sub info

\subsection{Second subpoint}\label{second-subpoint}

A haplotype block is a set of closely linked alleles or markers on a
chromosome that tend to be inherited together over evolutionary time.

Across Eukaryotes, the frequency of recombination is inversely
proportional to overall genome size. The result is that yeast have a
recombination rate a few orders of magnitude higher than that of humans
(She and Jarosz, 2018).

\section{DNA Replication}\label{dna-replication}

When the origin of replication(s) is removed from bacteria or
eukaryotes, growth and division is restricted or entirely eliminated,
but in some strains of archaea like H volcanii, deletion of the origin
of replication accelerates cell growth rates. It turns out that this
archaea can use a process that is similar to homologous recombination to
create a replication fork and replicate its chromosome (Hawkins et al.,
2013).

\section{Cloning}\label{cloning}

A macaque was the first primate to be cloned by SCNT (Liu et al., 2018).

\section{Mutation Rate}\label{mutation-rate}

Using whole-genome sequencing or next-gen sequencing to determine
mutation rates by base, it appears that C\textgreater{}T mutations at
CpG sites mutate at a frequency of 10-7 changes/cell division, and all
other sites are within the range of 10-8-10-9 base changes per division
(Arnheim and Calabrese, 2009, 2016; Campbell and Eichler, 2013; Ségurel
et al., 2014).

Providing an example of how human mutation rates can differ by
geographic origination, europeans compared to african/asian populations
have a 1.6 increased mutation rate of a particular mutation (Harris,
2015).

It appears that humans have the highest germline mutation rate of all
analyzed species (Lynch, 2016)

\section{Mutation Hotspots}\label{mutation-hotspots}

There are a number of sporadic mutation hotspots associated with disease
incidence, like achondroplasia which has a sporadic incidence rate of
4.5 x 10-5 per generation (Arnheim and Calabrese, 2016; Waller et al.,
2008). This disease originates from a single mutation in the FGFR3 gene
at a mutation rate \textasciitilde{}450 times higher than what would
ordinarily be expected at a CpG site (Bellus et al., 1995; Rousseau et
al., 1994; Shiang et al., 1994).

\section{Mutation Detection}\label{mutation-detection}

Pyrophosphorolysis-activated polymerization is a mutation detection
method that can detect a single mutant molecule of DNA within 25,000
genomes (Liu and Sommer, 2004; Qin et al., 2007).

\section{Genetic Modifications}\label{genetic-modifications}

Caffeine was cloned to allow for caffeine-deficient coffee and teas
without the decaffeination process (Kato et al., 2000).

When a DNA-associating protein from tardigrades was cloned into
mammalian cells, they became about 40\% more tolerant to radiation
(Hashimoto et al., 2016).

\section{Sequencing Methods}\label{sequencing-methods}

Using a new sequencing method called sci-RNA-seq, the transcriptome of
every cell of 762 cells in C Elegans was sequenced to yield single-cell
sequencing results and transcriptome profiling of every cell in the
body. The way this is done is by methanol fixing nuclei and then
incorporating a UMI when converting to cDNA, then mixing cells again and
incorporating another UMI when synthesizing the other strand (Cao et
al., 2017).

It appears that DAPI does not increase sequencing error rates by
Illumina sequencing (Leung et al., 2016).

One group came up with a method that is essentially identical to mine in
which they use barcoded probes to detect leukemia but they tracked the
mutation manually and ignored background (Wong et al. 2015)

\section{Diagnostics}\label{diagnostics}

In Li and Snyder Cell 2018, the EHR from hospitals is used to integrate
with a machine learning algorithm trained on aneurysm detection.
Patients are then whole genome sequenced, and the genome sequencing plus
the lifestyle of the individual on EHR is then used to predict if the
person has an aneurysm. They were able to achieve pretty robust
detection results that could then be used in a prediction setting in the
clinic.

\section{Detecting Common Diseases}\label{detecting-common-diseases}

Much of the following information comes from this review: (Shendure et
al. 2019) Linkage disequilibrium studies were designed to detect
Mendelian diseases GWAS designed back in 1996 to detect non-mendelian
multigenic traits that have much less penetrant effects The promise that
GWAS could risk stratify people for diseases has been challenging
because most diseases seem to be driven by an extremely large number of
variants with small effects that will likely require extremely large
sample sizes There exists a problem of missing heritability, and it was
often believed that common SNPs only held part of the puzzle, and more
rare variants accounted for a great deal of heritability, but this does
not yet seem to be the case, and SNPs seem to have a much greater effect
size Another problem with GWAS is it is haplotype specific in that it
can implicate a stretch of DNA inherited from one parent, but is blind
to the individual effect sizes of each of the individual variants A
challenge raised by Jonathan Pritchard is that gene regulatory networks
are so interconnected that variants in one gene may actually cause
changes in other genes and are therefore only peripherally relevant to a
phenotype One continued promise of the utility of GWAS to identify the
causal genetics behind diseases is that most of the strongest GWAS
associations came from small studies of european populations that
identified mutations of large effect sizes. By expanding studies to
populations, especially those like african populations that have less
linkage disequilibrium many more variants of large effect sizes could be
identified and used to tease out relationships of smaller effect sizes
in other populations. Methods are also improving fo linking regulatory
elements to the genes they regulate like (Gasperini et al. 2018;
Gasperini et al. 2019). Linking regulatory elements to their
corresponding genes can be quite helpful, because this information can
be incorporated into GWAS calculations to refine causal linkage
probabilities. Polygenic risk scores have often been used to predict
phenotypic variance in plants and animals, and have yet to really be
applied to human genomics (Khera et al. 2018). Training of PRSs seem to
not require fine-mapping, and their use has been aided by the UK Biobank
(Bycroft et al. 2018).

\section{Detecting Rare Diseases}\label{detecting-rare-diseases}

There are some 7k mendelian monogenic disorders that impact about 0.5\%
of live births, but contribute to about 70\% of pediatric hospital
admissions An important surprise has been that de-novo mutations account
for a substantial amount of intellectual disabilities and autism, where
as many as 30-60\% of ASD is caused by de-novo mutations Currently as
many as half of acutely ill inpatient infants can be diagnosed from WGS.
There are currently 59 genes designated by the American College of
Medical Genetics as being sufficiently clinically actionable as to
warrant sequencing and reporting in patients (Kalia et al. 2017).

\chapter{JupyterLab}\label{jupyter}

Here is a simple template that I use that controls a couple useful
things when starting a new notebook.

\begin{Shaded}
\begin{Highlighting}[]
\ImportTok{import}\NormalTok{ sys}
\NormalTok{sys.path.append(}\StringTok{'../util'}\NormalTok{)}

\OperatorTok\NormalTok{autoreload }\DecValTok{2}

\ImportTok{from}\NormalTok{ util }\ImportTok{import} \OperatorTok{*}
\ImportTok{import}\NormalTok{ numpy }\ImportTok{as}\NormalTok{ np                  }
\ImportTok{import}\NormalTok{ pandas }\ImportTok{as}\NormalTok{ pd                 }
\ImportTok{from}\NormalTok{ matplotlib }\ImportTok{import}\NormalTok{ pyplot }\ImportTok{as}\NormalTok{ plt}
\ImportTok{import}\NormalTok{ seaborn }\ImportTok{as}\NormalTok{ sns}

\NormalTok{sns.set_palette(}\StringTok{'pastel'}\NormalTok{)}
\NormalTok{sns.set_style(}\StringTok{'ticks'}\NormalTok{)}
\NormalTok{sns.set_context(}\StringTok{'paper'}\NormalTok{, font_scale}\OperatorTok{=}\DecValTok{1}\NormalTok{)}
\end{Highlighting}
\end{Shaded}

It is often convenient to have a notebook automatically refresh the
imported libraries so that they can be modified while working on a
JupyterLab notebook.

\begin{Shaded}
\begin{Highlighting}[]
\OperatorTok\NormalTok{autoreload }\DecValTok{2}
\end{Highlighting}
\end{Shaded}

To allow directory organization, dependcies can be separated into
different directories and imported into a jupyter notebook using the
following import statement.

\begin{Shaded}
\begin{Highlighting}[]
\ImportTok{import}\NormalTok{ sys}
\NormalTok{sys.path.append(}\StringTok{'../util'}\NormalTok{)}
\end{Highlighting}
\end{Shaded}

\chapter{Visualization}\label{visualization}

\section{Matplotlib}\label{matplotlib}

Plotting a heatmap.

\begin{Shaded}
\begin{Highlighting}[]
\ImportTok{import}\NormalTok{ matplotlib.pyplot }\ImportTok{as}\NormalTok{ plt}
\ImportTok{import}\NormalTok{ numpy }\ImportTok{as}\NormalTok{ np}
\NormalTok{a }\OperatorTok{=}\NormalTok{ np.random.random((}\DecValTok{16}\NormalTok{, }\DecValTok{16}\NormalTok{))}
\NormalTok{plt.imshow(a, cmap}\OperatorTok{=}\StringTok{'RdBu'', interpolation='}\NormalTok{nearest}\StringTok{')}
\StringTok{plt.show()}
\end{Highlighting}
\end{Shaded}

Possible heatmap colors are:

\begin{Shaded}
\begin{Highlighting}[]
\NormalTok{Accent, Accent_r, Blues, Blues_r, BrBG, BrBG_r, BuGn, BuGn_r, BuPu, BuPu_r, CMRmap, CMRmap_r, Dark2, Dark2_r, GnBu, GnBu_r, Greens, Greens_r, Greys, Greys_r, OrRd, OrRd_r, Oranges, Oranges_r, PRGn, PRGn_r, Paired, Paired_r, Pastel1, Pastel1_r, Pastel2, Pastel2_r, PiYG, PiYG_r, PuBu, PuBuGn, PuBuGn_r, PuBu_r, PuOr, PuOr_r, PuRd, PuRd_r, Purples, Purples_r, RdBu, RdBu_r, RdGy, RdGy_r, RdPu, RdPu_r, RdYlBu, RdYlBu_r, RdYlGn, RdYlGn_r, Reds, Reds_r, Set1,}
\NormalTok{Set1_r, Set2, Set2_r, Set3, Set3_r, Spectral, Spectral_r, Wistia, Wistia_r, YlGn, YlGnBu, YlGnBu_r, YlGn_r, YlOrBr, YlOrBr_r, YlOrRd, YlOrRd_r, afmhot, afmhot_r, autumn, autumn_r, binary, binary_r, bone, bone_r, brg, brg_r, bwr, bwr_r, cividis, cividis_r, cool, cool_r, coolwarm, coolwarm_r, copper, copper_r, cubehelix, cubehelix_r, flag, flag_r, gist_earth, gist_earth_r, gist_gray, gist_gray_r, gist_heat, gist_heat_r, gist_ncar, gist_ncar_r, gist_rainbow, gist_rainbow_r,}
\NormalTok{gist_stern, gist_stern_r, gist_yarg, gist_yarg_r, gnuplot, gnuplot2, gnuplot2_r, gnuplot_r, gray, gray_r, hot, hot_r, hsv, hsv_r, icefire, icefire_r, inferno, inferno_r, jet, jet_r, magma, magma_r, mako, mako_r, nipy_spectral, nipy_spectral_r, ocean, ocean_r, pink, pink_r, plasma, plasma_r, prism, prism_r, rainbow, rainbow_r, rocket, rocket_r, seismic, seismic_r, spring, spring_r, summer, summer_r, tab10, tab10_r, tab20, tab20_r, tab20b, tab20b_r, tab20c, tab20c_r, terrain, terrain_r,}
\NormalTok{twilight, twilight_r, twilight_shifted, twilight_shifted_r, viridis, viridis_r, vlag, vlag_r, winter, winter_r}
\end{Highlighting}
\end{Shaded}

A simple venn diagram.

\begin{Shaded}
\begin{Highlighting}[]
\ImportTok{from}\NormalTok{ matplotlib_venn }\ImportTok{import}\NormalTok{ venn2}
\NormalTok{venn2(subsets }\OperatorTok{=}\NormalTok{ (}\DecValTok{3}\NormalTok{, }\DecValTok{2}\NormalTok{, }\DecValTok{1}\NormalTok{))}
\end{Highlighting}
\end{Shaded}

A more complicated venn diagram.

\begin{Shaded}
\begin{Highlighting}[]
\ImportTok{from}\NormalTok{ matplotlib }\ImportTok{import}\NormalTok{ pyplot }\ImportTok{as}\NormalTok{ plt}
\ImportTok{import}\NormalTok{ numpy }\ImportTok{as}\NormalTok{ np}
\ImportTok{from}\NormalTok{ matplotlib_venn }\ImportTok{import}\NormalTok{ venn3, venn3_circles}
\NormalTok{plt.figure(figsize}\OperatorTok{=}\NormalTok{(}\DecValTok{4}\NormalTok{,}\DecValTok{4}\NormalTok{))}
\NormalTok{v }\OperatorTok{=}\NormalTok{ venn3(subsets}\OperatorTok{=}\NormalTok{(}\DecValTok{1}\NormalTok{, }\DecValTok{1}\NormalTok{, }\DecValTok{1}\NormalTok{, }\DecValTok{1}\NormalTok{, }\DecValTok{1}\NormalTok{, }\DecValTok{1}\NormalTok{, }\DecValTok{1}\NormalTok{), set_labels }\OperatorTok{=}\NormalTok{ (}\StringTok{'A'}\NormalTok{, }\StringTok{'B'}\NormalTok{, }\StringTok{'C'}\NormalTok{))}
\NormalTok{v.get_patch_by_id(}\StringTok{'100'}\NormalTok{).set_alpha(}\FloatTok{1.0}\NormalTok{)}
\NormalTok{v.get_patch_by_id(}\StringTok{'100'}\NormalTok{).set_color(}\StringTok{'white'}\NormalTok{)}
\NormalTok{v.get_label_by_id(}\StringTok{'100'}\NormalTok{).set_text(}\StringTok{'Unknown'}\NormalTok{)}
\NormalTok{v.get_label_by_id(}\StringTok{'A'}\NormalTok{).set_text(}\StringTok{'Set "A"'}\NormalTok{)}
\NormalTok{c }\OperatorTok{=}\NormalTok{ venn3_circles(subsets}\OperatorTok{=}\NormalTok{(}\DecValTok{1}\NormalTok{, }\DecValTok{1}\NormalTok{, }\DecValTok{1}\NormalTok{, }\DecValTok{1}\NormalTok{, }\DecValTok{1}\NormalTok{, }\DecValTok{1}\NormalTok{, }\DecValTok{1}\NormalTok{), linestyle}\OperatorTok{=}\StringTok{'dotted'}\NormalTok{)}
\NormalTok{c[}\DecValTok{0}\NormalTok{].set_lw(}\FloatTok{1.0}\NormalTok{)}
\NormalTok{c[}\DecValTok{0}\NormalTok{].set_ls(}\StringTok{'dotted'}\NormalTok{)}
\NormalTok{plt.title(}\StringTok{"Sample Venn diagram"}\NormalTok{)}
\NormalTok{plt.annotate(}\StringTok{'Unknown set'}\NormalTok{, xy}\OperatorTok{=}\NormalTok{v.get_label_by_id(}\StringTok{'100'}\NormalTok{).get_position() }\OperatorTok{-}\NormalTok{ np.array([}\DecValTok{0}\NormalTok{, }\FloatTok{0.05}\NormalTok{]), xytext}\OperatorTok{=}\NormalTok{(}\OperatorTok{-}\DecValTok{70}\NormalTok{,}\OperatorTok{-}\DecValTok{70}\NormalTok{),}
\NormalTok{             ha}\OperatorTok{=}\StringTok{'center'}\NormalTok{, textcoords}\OperatorTok{=}\StringTok{'offset points'}\NormalTok{, bbox}\OperatorTok{=}\BuiltInTok{dict}\NormalTok{(boxstyle}\OperatorTok{=}\StringTok{'round,pad=0.5'}\NormalTok{, fc}\OperatorTok{=}\StringTok{'gray'}\NormalTok{, alpha}\OperatorTok{=}\FloatTok{0.1}\NormalTok{),}
\NormalTok{                          arrowprops}\OperatorTok{=}\BuiltInTok{dict}\NormalTok{(arrowstyle}\OperatorTok{=}\StringTok{'->'}\NormalTok{, connectionstyle}\OperatorTok{=}\StringTok{'arc3,rad=0.5'}\NormalTok{,color}\OperatorTok{=}\StringTok{'gray'}\NormalTok{))}
\NormalTok{                          plt.show()}
\end{Highlighting}
\end{Shaded}

\section{Seaborn}\label{seaborn}

Here is a general bar plot that includes some commonly used parameters.

\begin{Shaded}
\begin{Highlighting}[]
\CommentTok{# fits my 22 inch monitor}
\NormalTok{plt.figure(figsize}\OperatorTok{=}\NormalTok{(}\FloatTok{19.17}\NormalTok{,}\FloatTok{11.98}\NormalTok{))}
\CommentTok{# order controls the display order of the samples}
\NormalTok{sns.catplot(x}\OperatorTok{=}\StringTok{"Sample"}\NormalTok{, y}\OperatorTok{=}\StringTok{"Somatic"}\NormalTok{, kind}\OperatorTok{=}\StringTok{"bar"}\NormalTok{, data}\OperatorTok{=}\NormalTok{var_counts, order}\OperatorTok{=}\NormalTok{labels)}\OperatorTok{;}
\CommentTok{# keeps x-axis labels, but eliminates the tick mark}
\NormalTok{plt.tick_params(labelbottom}\OperatorTok{=}\VariableTok{True}\NormalTok{, bottom}\OperatorTok{=}\VariableTok{False}\NormalTok{)}
\CommentTok{# trim off the x-axis}
\NormalTok{sns.despine(offset}\OperatorTok{=}\DecValTok{10}\NormalTok{, trim}\OperatorTok{=}\VariableTok{True}\NormalTok{, bottom}\OperatorTok{=}\VariableTok{True}\NormalTok{)}
\CommentTok{# labels}
\NormalTok{plt.title(}\StringTok{''}\NormalTok{)}
\NormalTok{plt.ylabel(}\StringTok{''}\NormalTok{)}
\NormalTok{plt.xlabel(}\StringTok{''}\NormalTok{)}
\CommentTok{# manual control of xlabels}
\NormalTok{labels }\OperatorTok{=}\NormalTok{ [}\StringTok{'Indiv_1-a'}\NormalTok{,}\StringTok{'Indiv_2'}\NormalTok{,}\StringTok{'Indiv_3'}\NormalTok{,}\StringTok{'Indiv_1-b'}\NormalTok{]}
\CommentTok{# control xtick order}
\NormalTok{plt.xticks(}\BuiltInTok{range}\NormalTok{(}\BuiltInTok{len}\NormalTok{(labels)), labels, rotation}\OperatorTok{=}\DecValTok{45}\NormalTok{)}
\CommentTok{# control the number of x-ticks}
\NormalTok{plt.locator_params(axis}\OperatorTok{=}\StringTok{'x'}\NormalTok{, nbins}\OperatorTok{=}\DecValTok{10}\NormalTok{)}
\CommentTok{# legend positioning}
\NormalTok{plt.legend(loc}\OperatorTok{=}\StringTok{'upper right'}\NormalTok{)}
\CommentTok{# log scale}
\NormalTok{plt.gca().set_yscale(}\StringTok{'log'}\NormalTok{)}
\CommentTok{# this is better if neg values are needed}
\NormalTok{plt.gca().set_yscale(}\StringTok{'symlog'}\NormalTok{)}
\CommentTok{# fit plot to display}
\NormalTok{plt.tight_layout()}
\NormalTok{plt.show()}
\CommentTok{# save figure with tight_layout}
\NormalTok{plt.savefig(}\StringTok{"test.svg"}\NormalTok{, }\BuiltInTok{format}\OperatorTok{=}\StringTok{"svg"}\NormalTok{, bbox_inches}\OperatorTok{=}\StringTok{"tight"}\NormalTok{)}
\end{Highlighting}
\end{Shaded}

Signifance information can be added by including p-values and label bars
using the following code.

\begin{Shaded}
\begin{Highlighting}[]
\NormalTok{x1, x2 }\OperatorTok{=} \DecValTok{0}\NormalTok{, }\DecValTok{1} \CommentTok{# columns to annotate on the plot}
\NormalTok{y2, y1 }\OperatorTok{=} \DecValTok{20}\NormalTok{, }\DecValTok{15} \CommentTok{# placement of the line and how for down the vertical legs go}
\NormalTok{plt.plot([x1,x1, x2, x2], [y1, y2, y2, y1], linewidth}\OperatorTok{=}\DecValTok{1}\NormalTok{, color}\OperatorTok{=}\StringTok{'k'}\NormalTok{) }\CommentTok{# stats line}
\NormalTok{plt.text((x1}\OperatorTok{+}\NormalTok{x2)}\OperatorTok{*}\NormalTok{.}\DecValTok{5}\NormalTok{, y2}\OperatorTok{+}\DecValTok{2}\NormalTok{, }\StringTok{"p=0.09"}\NormalTok{, ha}\OperatorTok{=}\StringTok{'center'}\NormalTok{, va}\OperatorTok{=}\StringTok{'bottom'}\NormalTok{) }\CommentTok{# p-value or sig}
\end{Highlighting}
\end{Shaded}

\chapter{Biology}\label{biology}

\section{General}\label{general}

Some helpful commands for genetic sequence.

\begin{Shaded}
\begin{Highlighting}[]
\ImportTok{from}\NormalTok{ string }\ImportTok{import}\NormalTok{ ascii_uppercase }\CommentTok{# python 3}
\ImportTok{from}\NormalTok{ string }\ImportTok{import}\NormalTok{ upper, lower }\CommentTok{# python 2}
\NormalTok{upper(}\StringTok{'tcga'}\NormalTok{)}
\NormalTok{lower(}\StringTok{'TCGA'}\NormalTok{)}
\NormalTok{title(}\StringTok{'tcga'}\NormalTok{) }\CommentTok{# capitalize the first letter}
\end{Highlighting}
\end{Shaded}

\section{Biopython}\label{biopython}

Reverse complement of sequence

\begin{Shaded}
\begin{Highlighting}[]
\ImportTok{from}\NormalTok{ Bio.Seq }\ImportTok{import}\NormalTok{ Seq}
\BuiltInTok{str}\NormalTok{(Seq(i).reverse_complement())}
\end{Highlighting}
\end{Shaded}

\section{UCSC Genome Browser}\label{ucsc-genome-browser}

Get sequence from UCSC genome browser

\begin{Shaded}
\begin{Highlighting}[]
\ImportTok{from}\NormalTok{ subprocess }\ImportTok{import}\NormalTok{ check_output, STDOUT}
\NormalTok{temp }\OperatorTok{=}\NormalTok{ check_output(}\StringTok{'wget -qO- http://genome.ucsc.edu/cgi-bin/das/hg19/dna?segment=}\SpecialCharTok{%s}\StringTok{:}\SpecialCharTok{%s}\StringTok{,}\SpecialCharTok\NormalTok{ (vcfObj.chrom,low,high), stderr}\OperatorTok{=}\NormalTok{STDOUT, shell}\OperatorTok{=}\VariableTok{True}\NormalTok{)}
\end{Highlighting}
\end{Shaded}

\section{Ref Genome}\label{ref-genome}

Get sequence from reference genome

\begin{Shaded}
\begin{Highlighting}[]
\ImportTok{from}\NormalTok{ subprocess }\ImportTok{import}\NormalTok{ check_output, STDOUT}
\NormalTok{temp }\OperatorTok{=}\NormalTok{ check_output(}\StringTok{'samtools faidx }\SpecialCharTok{%s}\StringTok{ }\SpecialCharTok{%s}\StringTok{:}\SpecialCharTok{%s}\StringTok{-}\SpecialCharTok\NormalTok{ (ref, vcfObj.chrom, low, high), stderr}\OperatorTok{=}\NormalTok{STDOUT, shell}\OperatorTok{=}\VariableTok{True}\NormalTok{)}

\NormalTok{finalSeq }\OperatorTok{=} \StringTok{''}
\ControlFlowTok{for}\NormalTok{ line }\KeywordTok{in}\NormalTok{ temp.decode(}\StringTok{'UTF-8'}\NormalTok{).split(}\StringTok{'}\CharTok{\textbackslash{}n}\StringTok{'}\NormalTok{):}
\ControlFlowTok{for}\NormalTok{ line }\KeywordTok{in}\NormalTok{ temp.decode(}\StringTok{'UTF-8'}\NormalTok{).split(}\StringTok{'}\CharTok{\textbackslash{}n}\StringTok{'}\NormalTok{): }\CommentTok{# this is only necessary in python 3 to convert binary to string}
    \ControlFlowTok{if} \StringTok{'>'} \KeywordTok{not} \KeywordTok{in}\NormalTok{ line:}
\NormalTok{        finalSeq }\OperatorTok{+=}\NormalTok{ line}

\NormalTok{finalSeq }\OperatorTok{=}\NormalTok{ finalSeq.upper()}
\end{Highlighting}
\end{Shaded}

\section{Personal Information}\label{personal-information}

\begin{Shaded}
\begin{Highlighting}[]
\CommentTok{# parse vcf file with parseline}
\ControlFlowTok{if} \StringTok{'#'} \KeywordTok{not} \KeywordTok{in}\NormalTok{ line }\KeywordTok{and} \StringTok{'chr'} \KeywordTok{in}\NormalTok{ line: }\CommentTok{# skip the info}
\CommentTok{# vcf handling}
\ImportTok{from}\NormalTok{ parseline }\ImportTok{import}\NormalTok{ VCFObj}
\CommentTok{# or}
\ImportTok{from}\NormalTok{ util }\ImportTok{import}\NormalTok{ VCFObj}
\NormalTok{vcfObj }\OperatorTok{=}\NormalTok{ VCFObj(vcfLine)}
\CommentTok{# available attributes: ao, dp, af, wt, var, chrom, location}
\end{Highlighting}
\end{Shaded}

\chapter{Data I/O}\label{io}

\section{Reading Data Files}\label{reading-data-files}

Opening .gz files

\begin{Shaded}
\begin{Highlighting}[]
\ImportTok{import}\NormalTok{ gzip}
\ControlFlowTok{for}\NormalTok{ line }\KeywordTok{in}\NormalTok{ gzip.}\BuiltInTok{open}\NormalTok{(}\StringTok{'myFile.gz'}\NormalTok{):}
    \BuiltInTok{print}\NormalTok{ line}
\end{Highlighting}
\end{Shaded}

\section{Pickles}\label{pickles}

Writing data in pickle format

\begin{Shaded}
\begin{Highlighting}[]
\ImportTok{import}\NormalTok{ pickle}
\NormalTok{p }\OperatorTok{=} \BuiltInTok{open}\NormalTok{(}\StringTok{'principle.pkl'}\NormalTok{, }\StringTok{'wb'}\NormalTok{)}
\NormalTok{pickle.dump(principleData, p)}
\NormalTok{p.close()}
\end{Highlighting}
\end{Shaded}

Reading data in pickle format

\begin{Shaded}
\begin{Highlighting}[]
\ImportTok{import}\NormalTok{ pickle}
\NormalTok{p }\OperatorTok{=} \BuiltInTok{open}\NormalTok{(}\StringTok{'principle.pkl'}\NormalTok{, }\StringTok{'rb'}\NormalTok{)}
\NormalTok{principleData }\OperatorTok{=}\NormalTok{ pickle.load(p)}
\NormalTok{p.close()}
\end{Highlighting}
\end{Shaded}

\bibliography{book.bib,packages.bib}


\end{document}
